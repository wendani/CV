% resume.tex
%
% (c) 2002 Matthew Boedicker <mboedick@mboedick.org> (original author) http://mboedick.org
% (c) 2003 David J. Grant <dgrant@ieee.org> http://www.davidgrant.ca
% (c) 2007 Todd C. Miller <Todd.Miller@courtesan.com> http://www.courtesan.com/todd
%
%This work is licensed under the Creative Commons Attribution-NonCommercial-ShareAlike License. To view a copy of this license, visit http://creativecommons.org/licenses/by-nc-sa/1.0/ or send a letter to Creative Commons, 559 Nathan Abbott Way, Stanford, California 94305, USA.

\documentclass[letterpaper,11pt]{article}
%\documentclass[conference,onecolumn]{IEEEtran}

%\setlength{\parindent}{0in}

%-----------------------------------------------------------
\usepackage[empty]{fullpage}
\usepackage[usenames]{color}
%\usepackage{eurosym}
\usepackage{marvosym}
\usepackage{enumitem}
\usepackage{multirow}
\usepackage{url}
\usepackage{romannum}
%\usepackage[T1]{fontenc}
%\usepackage{path}
%\usepackage{hyperref}
\usepackage{textcomp}
\definecolor{mygrey}{gray}{0.80}
\textheight=9.0in
%\raggedbottom
%\raggedright
\setlength{\tabcolsep}{0in}

% Adjust margins
\addtolength{\oddsidemargin}{-0.375in}
\addtolength{\evensidemargin}{0.375in}
\addtolength{\textwidth}{0.5in}
\addtolength{\topmargin}{-.375in}
\addtolength{\textheight}{0.75in}

\setlength{\parindent}{0pt} % no indent at the beginning of each paragraph
%\setlist{leftmargin=0.35in}

%Custom commands
\newcommand{\resheading}[1]{{\noindent\large \colorbox{mygrey}{
\begin{minipage}{1.0\textwidth}{\textsc{#1 \vphantom{p\^{E}}}}\end{minipage}}}}
% 1.01\textwidth

\newcommand{\ressubheading}[4]{
\begin{tabular*}{6.69in}{l@{\extracolsep{\fill}}r}
        \textbf{#1} & #2 \\
        \textit{#3} & \textit{#4} \\
\end{tabular*}\vspace{-6pt}
}

\newcommand{\resitem}[1]{\item #1 \vspace{-2pt}}





\begin{document}


\textbf{\large Wenda Ni, Ph.D.}\\
\rule [1ex] {1.0\linewidth} {1pt} %\\%
%%%
%%%%%%%%%%%%%%%%%%%%%%%%%%%%%%%%%%%%%%%%%%%%%%%%%%%%%%%%%%%%%%%%%%%%%%%%%%%%%%%%%%%%%%%%%%%%%%%%%%%%%%%%%%
%\begin{figure}[!h]
%\setlength{\abovecaptionskip}{0pt}
%\setlength{\belowcaptionskip}{0pt}%
%%\centering
%
%\begin{minipage}[]{0.35\linewidth}%%%%%%%%%%%%%%%%%%%%%%%%%%%%%%%%%%%%%%%%%%%%%%%%
\begin{tabular*}{7in}{l@{\extracolsep{\fill}}r}
9839 NE 138th PL&  \\
Kirkland, WA, 98034  & M: +1 (425) 984 9332\\
United States & E: wonda.ni@gmail.com\\
\end{tabular*}
%\end{minipage}%%%%%%%%%%%%%%%%%%%%%%%%%%%%%%%%%%%%%%%%%%%%%%%%%%%%%%%%%%%%%%%%%%%%%%
%%\hspace{0.5ex} %
%%\begin{minipage}[]{0.65\linewidth}%%%%%%%%%%%%%%%%%%%%%%%%%%%%%%%%%%%%%%%%%%%%%%%%%%
%%\end{minipage}%%%%%%%%%%%%%%%%%%%%%%%%%%%%%%%%%%%%%%%%%%%%%%%%%%%%%%%%%%%%%%%%%%%%%%
%\end{figure}


\vspace{0.1in}
\resheading{Education}%%%%%%%%%%%%%%%%%%%%%%%%%%%%%%%%%%%%%%%%%%%%%%%%%%%%%%%%%%%

\begin{itemize}% [leftmargin=0.35in]
\setlength{\itemindent}{-0.075in}

\item
    \ressubheading{Tsinghua University}{Beijing, P. R. China}
    {Ph.D., Dept. of Electronic Engineering}{Sept. 2005 - Jul. 2010} %
    \begin{itemize}
        %\resitem{Research interests: survivability, service differentiation, and multi-layer traffic engineering in backbone networks}
        \resitem{Topics: Network design, operation, and management with a focus on survivability, service differentiation,
	and two-layer traffic engineering in telecom transport networks}
        \resitem{Thesis: Resource optimization and service quality in WDM optical networks}
    \end{itemize}

\item
    \ressubheading{Tsinghua University}{Beijing, P. R. China}
    {Bachelor Degree of Engineering, Dept. of Electronic Engineering}{Sept. 2001 - Jul. 2005}%
    \begin{itemize}
        \resitem{Major: Physical electronics and Optoelectronics}
        \resitem{GPA: 88.8/100 Rank: 5/80}
	%\resitem{\textcolor{Red}{Excellent graduate, an honor entitled to top 10\% graduates}}
    \end{itemize}

\end{itemize}




%\leftskip 0.0cm
%\vspace{0.05in}
\resheading{Work Experience}%%%%%%%%%%%%%%%%%%%%%%%%%%%%%%%%%%%%%%%%%%%%%%%%%%%%%%%%
\begin{itemize}
\setlength{\itemindent}{-0.075in}

\item
  \ressubheading{Microsoft Corporation}{Redmond, WA, USA}
  {Software engineer \Romannum{2}, Azure Networking}{Jan. 2019 - present}
  \begin{itemize}
  \resitem{Topics: SONiC network operating system over world's bleeding-edge datacenter switching chips and switches (40G, 50G, 100G);
  RDMA over SONiC with full engineering responsbility of all RDMA-related components---PFC (DSCP-based and DOT1P-based), PFC storm watchdog,
  memory management unit (MMU), buffer watermark, ECN marking,
  WRED dropping, QoS mapping, and PFC storm watchdog warm-reboot}
  \resitem{Designed, implemented, and test automated PFC watchdog warm-reboot feature, with ramp-up on the
  warm-reboot architecture from scratch.
  Resolved the issue of PFC watchdog not started after (both cold- and warm-) reboot by fixing the initialization dependancy.}
  \resitem{Executed full development cycle of buffer pool watermark feature.
  Improved the scalability of flex counter polling infrastructure with full sleep wait.
  Established synchronous counter clear path all the way down to ASIC to probe the hardware capability.}
  \end{itemize}

\item
  \ressubheading{Microsoft Corporation}{Vancouver, BC, Canada}
  {Software engineer 2, Azure Networking}{Dec. 2017 - Jan. 2019}
  \begin{itemize}
  \resitem{Topics: white-box datacenter switches (40G, 50G, 100G)}
  \resitem{Developed fundamental insights into the memory management, scheduler, QoS, and flow control units of Broadcom XGS devices
  through theory of operation study and hand-on experiements with various chips.}
  \resitem{Characterized and test automated the DSCP mapping, ingress admission control (flow control XOFF/XON threshold, pool size),
  egress admission control (drop state, pool size), ECN marking, and WRED dropping of MMU
  on Broadcom BCM56850, BCM56960, BCM56970, and Mellanox Spectrum switch chips
  for various traffic types (lossy, lossless, data, and CPU).
  While being open-source, this part of the codes remains proprietary due to its intensity in design and methodology innovation.}
  \resitem{Designed, implemented, and validated WRED/ECN on/off feature within two weeks to elminate the production rollout blocker.
  Started from zero coverage of the SONiC code base, got to the heart of the codes, and got the feature done within a stringent timeline.}
  \resitem{Built a solid foundation on Broadcom XGS switch chip pipeline from an exhaustive study of Broadcom University training materials
  and hand-on practice.}
  \end{itemize}

\item
  \ressubheading{Pleora Technologies Inc.}{Ottawa, ON, Canada}
  {Senior embedded software designer}{Jan. 2017 - Dec. 2017}
  \begin{itemize}
    \resitem{Topics: real-time video backhaul over GigE Vision standard protocol suite}
    \resitem{Designed and developed embedded software in a C and C++ mixed environment. Summoned by the core project team within six months since onboarding.
    Accomplished (passed product verification) five features in C++ with design pattern application in the first month of joining the core project.}
    \resitem{Improved the USRT (serial port communication) latency performance by over six times through batch data push and bug fixing.
    Identified architectural bottleneck that prohibits further performance improvements.}
    \resitem{Grasped the agent-side SNMP implementation in depth by learning the SNMP fundamentals from scratch
    and leading through the code base. Pinpointed and fixed the bug within a week.}
    \resitem{Ported third-party SNTPv4 module into our network stack and interfaced it with our embedded software for configuration, control, and application.
    Spotted bug (confirmed by third party) in code reading on the third day of taking over the source code.}
    \resitem{Proposed a lazy address conflict resolution method to reduce device boot-up time by $\sim$7 seconds.
    Verified the feasibility of the method via experimental trials.
    Proposal adopted and implemented in Pleora's latest devices for boot-up time optimization.}
    %by skipping ARP probing (defined in RFC 5227 IPv4 address conflict detection) in IP configuration and
    %deferring the address conflicts to manual resolution over GigE Vision FORCEIP control packet;
    \resitem{Reduced the completion time of SNMP Get- and Set-Request by up to over 15\% through agent-side MIB lookup acceleration using binary search.}
  \end{itemize}

\item
  \ressubheading{Viscore Technologies Inc. (defunct)}{Ottawa, ON, Canada}
  {Principal R\&D engineer on network architecture, software, and hardware}{Sept. 2014 - Dec. 2016}
  \begin{itemize}
    \resitem{Topics: reliable data replication over optical broadcast medium (optical coupler); prototype of HEDA protocol on NetFPGA-10G board,
    funded in part by OCE TalentEdge fellowship; low-latency data communication over RDMA}
    \resitem{Setup NetFPGA-10G board on Fedora 14 through package installation, kernel module loading, and troubleshooting.}
    \resitem{Demonstrated UDP unicast, broadcast, and multicast over optical coupler. The uniqueness of the system is unidirectional communication links.}
    \resitem{Grasped in-depth knowledge of NIC hardware data plane and control plane architecture and of the role each module plays
    with respect to Ethernet physical layer and link layer standards.
    Enabled burst-mode transmission---a key foundation of communication over an optical coupler---at the physical layer
    through disabling link fault signaling at the RS sublayer and masking fault alarms at the PCS sublayer.}
    \resitem{Developed a control-plane method to perform burst transmission---an enabling technology of prototyping HEDA protocol on NetFPGA-10G board.
    Designed and implemented a user-space token-passing protocol in C to demonstrate burst data transmission between two hosts over an optical coupler.
    This milestone eliminates the biggest challenge to prototype the HEDA protocol.}
    \resitem{Learned exhaustive knowledge of Linux kernel architecture, Linux kernel programming, Linux network device driver, and Linux network stack.
    Architected the integration and interface solution of embracing HEDA protocol in the Linux networking system (kernel 2.6.35.14).
    Designed and implementing software modules of HEDA protocol in C at the kernel network device driver level.}
    \resitem{Led a team of three at the hardware development phase and a team of two at the software development phase.}
    \resitem{Acquired a deep understanding of various RDMA technologies---InfiniBand, RoCE, and iWARP---through
    standard study (IBTA specification and IETF RFCs), programming practice, and hands-on experience playing with various NICs on our platform.
    Refactored OFA (OpenFabrics Alliance) libibverbs example code to work with Mellanox OFED.}
    \resitem{Gained insights into the internal workings of RoCE data path and control path through code reading of open source Soft-RoCE implementation.}
    \resitem{Peeled Soft-RoCEv2 off Linux source tree, and ported it to work with kernel 4.2.0-rc8 as a kernel module.}
  \end{itemize}


\item
  \ressubheading{Source Technologies International, Ltd.}{Ottawa, ON, Canada}
  {Consulting software engineer}{Jun. 2014 - Aug. 2014}
  \begin{itemize}
    \resitem{Developed a mobile scenario manager app---a commercial product for iPhone iOS---in 1.5 months with both Objective-C and iOS programming learned from scratch.
    The key techniques employed include UIStoryboard, Model-View-Controller (MVC) pattern, target-action pattern, delegation pattern, singleton pattern,
    core data framework to interact with a local SQLite database, JSON text parsing,
    and HTTP/HTTPS POST method to communicate with a remote server for credential authentication and scenario fetching, update, and launching.}
    \resitem{Tested app on iOS simulator and real iPhone device through Xcode. Created and Distributed iOS App Store Package to testers and clients for beta testing.}
    \resitem{Developed an Android counterpart in Java fulfilling the same requirements with Android programming learned from scratch.
    Gained insights into Activity/Fragment lifecycle, and managed state transitions through callback methods.
    Designed user interface layout using XML. Debugged Java code using LogCat.}
  \end{itemize}

\item
    \ressubheading{NEC Laboratories America, Inc.}{Princeton, NJ, USA}%
    {Visiting researcher, Dept. of optical networking}{Jan. 2011 - Feb. 2011}
    \begin{itemize}
        \resitem{Topic: Xen hypervisor and virtual machine (VM) live migration}
        \resitem{Setup Xen 4.0.1 pv-ops Ubuntu 10.10 through kernel configuration \& building and kernel module building \& installation/loading.
        Setup a network file system server to store the disk image that can be mounted concurrently by multiple VM clients.
        Demonstrated VM live migration between two physical hosts connected by a D-Link gigabit switch (DGS-2205).}
    \end{itemize}

\item
    \ressubheading{Ericsson (China) Communications Co., Ltd.}{Bejing, P. R. China}%
    {System design engineer intern, Dept. of broadband networks}{Jul. 2009 - Sept. 2009}
    \begin{itemize}
        \resitem{Topic: IEEE 1588v2 (precision clock synchronization protocol) over Ethernet}
        \resitem{Developed a solid understanding of the clock synchronization mechanism. Finished one technical report on protocol review.
        The knowledge acquired becomes critical in later designing a link-layer polling protocol for optical datacenters (\textcolor{Red}{project with Viscore Technologies Inc., Canada}),
        where the issue of clock synchronization among ports was resolved using a similar yet simpler scheme.}
    \end{itemize}

\end{itemize}



%\leftskip 0.0cm
%\vspace{0.1in}
\resheading{R\&D Experience}%%%%%%%%%%%%%%%%%%%%%%%%%%%%%%%%%%%%%%%%%%%%%%%%%%%%%%%

\begin{itemize}% [leftmargin=0.35in]
\setlength{\itemindent}{-0.075in}

\item
    \ressubheading{Carleton University}{Ottawa, ON, Canada} %
    {Senior researcher, Dept. of Systems and Computer Engineering}{Jun. 2011 - Aug. 2014} %Postdoctoral fellow
    \begin{itemize}
        \resitem{Topics: 1) Datacenter networking; 2) Optical datacenters; 3) Protocol design; 4) Survivable telecom network design and analysis;
        5) Decomposition methods, in particular, Lagrangian relaxation and column generation, for solving network optimization problems; 6) Software-defined networks;
        7) Passive optical networks; 8) Optical node architecture}
        \resitem{Designed optimal link capacity to enable full bandwidth communication in emerging datacenter networks using fat-tree and Microsoft VL2 topologies}
        \resitem{Proposed a low-cost, power-efficient, and reliable datacenter network architecture using passive optical devices.
        Resolved physical-layer scalability issues by employing advanced interconnection techniques.
        Designed a fully distributed link-layer communication protocol, termed HEDA, to enable collision-free frame transmission.
        Described the protocol using finite state machine and message sequence chart.
        Developed an analytical model to compute lower and upper bounds on the expected packet delay of our protocol.
        (\textcolor{Red}{This line of achievements are outputs of a six-month industry research project with Viscore Technologies Inc., Canada.)}}
        \resitem{Studied a novel layered network architecture---Valiant load balancing (VLB) networks over optical networks.
        Developed a network-level availability model to compute the probability that a VLB network is congestion-free under all traffic patterns.
        The main challenges in such a context arise from the unique routing and protection scheme that goes beyond the definition of conventional connection-level service availability
        as well as the logical link failure correlation that prohibits the use of traditional analytical methods.
        \textcolor{Red}{Our work was honored as a runner-up of the 2013 Fabio Neri Best Paper Award.}
        One winner and three runners-up were selected from all papers published in \textit{Elsevier Optical Switching and Networking} journal throughout year 2013.
        Official announcement: \url{http://www.journals.elsevier.com/optical-switching-and-networking/awards/2013-fabio-neri-best-paper-award/}}

%        \vspace{-0.18in}
        \resitem{Developed analytical models to accurately compute the availability of upper-layer connections in two-layer networks,
        where dedicated path protection is deployed at either the lower layer or the upper layer.
        The accuracy of our model is validated through OPNET simulation. A real-life case of such two-layer networks is IP over optical networks.}
%        \resitem{Learned to be versatile on various research topics}
        \resitem{Learned to be versatile on various network topics. Developed mental toughness in demanding R\&D environments.}
    \end{itemize}

\item
    \ressubheading{State University of New York (SUNY) at Buffalo}{Buffalo, NY, USA} %
    {Senior research scientist, Dept. of Computer Science and Engineering}{Oct. 2010 - May 2011} %
    \begin{itemize}
        \resitem{Topics: network function virtualization; virtual network embedding; reliable network design and analysis}
        \resitem{Studied service availability in virtual infrastructure mapping and cloud computing such as Amazon EC2.
        Implemented a numerical method in MATLAB to compute the probability that the target availability is not fulfilled over a finite service subscription period.
        Since the service duration is finite, conventional steady-state analysis cannot be applied.
        Rather, the numerical calculation is a method that performs transient analysis.
        Our study can help service providers to work out service level agreements that avoid significant penalty risk.}
        \resitem{Learned to get along with supervisors and managers}
    \end{itemize}

\item
    \ressubheading{Fraunhofer Heinrich-Hertz-Institut (HHI)}{Berlin, Germany}% Fraunhofer-Institute for Telecommunications,
    {Visiting student, Dept. of Photonic Networks and Systems}{Oct. 2008 - Mar. 2009} %
    \begin{itemize}
        \resitem{Topic: survivability of WDM networks}
        \resitem{Evaluated connection availability in optical networks that employ shared path protection and backup path reprovisioning after failure occurrence.
        Compared the impact of revertive and non-revertive operations after failure repair on connection availability.}
        \resitem{Proposed and studied a new logical topology mapping problem, which incorporates the lightpath-based physical-layer failure localization capability
        into the traditional mapping design of IP over transparent optical networks.
        \textcolor{Red}{I, as the first author, was selected as a semi-finalist in the 2010 Corning Outstanding Student Paper Competition for this work.}
        Over 430 student submissions were received that year, of which 10-12 papers were selected as semi-finalists.}
    \end{itemize}

\item
    \ressubheading{Dept. of Electronic Engineering, Tsinghua University}{Beijing, P. R. China}%
    {Research assistant, Lab. of Optical Networking and Microwave Photonics}{Feb. 2005 - Jul. 2010} %
    \begin{itemize}
        \resitem{Kept updated with the cutting-edge research in optical networking}
        \resitem{Responsible for project implementation, including topology and routing algorithm design and optimization,
        differentiated service provisioning, protection and restoration, service availability analysis, multi-period network planning, etc.}
        \resitem{Carried out joint research work with peers} %(e.g., Michael Schlosser, Heinrich-Hertz-Institut, Germany; Dr. Yabin Ye, Huawei, Germany)}
        \resitem{Took major roles in drafting research proposals based on the industry trends
        (proposal ``Resource allocation and optimization in optical networks'' funded by National 863 High Tech Program, year 2007-2008; %, grant no. 2006AA01Z249
        proposal ``Path-computation-element-based (PCE-based) network architecture and testbed for multi-layer multi-domain transport
        networks'' funded by National 863 High Tech Program, year 2009-2010)} %, grant no. 2009AA01Z254
        \resitem{Networked with world-class research groups and scientists through conferences and emails} %(e.g., Dr. Victor Yu Liu, Juniper Networks, USA; Prof.-Dr. Jing Wu, CRC, Canada; Dr. Gangxiang Shen, Ciena, USA; Prof.-Dr. Darli A. A. Mello, University of Brasilia; Prof.-Dr. Biswanath Mukherjee, UCDavis; Dr. Dominic A. Schupke, Nokia Siemens Networks, Germany; etc.)}
    \end{itemize}

\end{itemize}




\resheading{Computer Skills}%%%%%%%%%%%%%%%%%%%%%%%%%%%%%%%%%%%%%%%%%%%%%%%%%%%%%%%%%%

\begin{description}
\setlength{\itemindent}{-0.1in}

\item[Programming Languages:] C ({\tt const}, {\tt volatile}, strstr, \%z), C++ in C, assembly (Nios II, x86 \& x64 (Intel \& AT\&T syntaxes), MicroBlaze, ARM),
inline assembly (gcc), C in C++,
%
C++ ({\tt mutable}, {\tt struct}, constructor (top-down, inside out), member initializer list, {\tt explicit}, direct initialization, copy initialization,
destructor (bottom-up, outside in), address of reference, pointer to member function, operator(), operator type,
cast ({\tt dynamic\_cast}), {\tt swap},
implicit interface, virtual function \& default argument, {\tt initializer\_list},
%
sequential container (iterator, {\tt emplace}, {\tt insert}, const\_iterator,
{\tt empty}, {\tt size}, operator[ ], {\tt at}, operator=, {\tt front}, {\tt emplace\_back},
{\tt push\_back}, {\tt pop\_back}, {\tt erase}, {\tt clear}),
%
string ({\tt NULL}, rational operators, {\tt find}, {\tt substr}, {\tt stoul}, {\tt npos}),
vector ({\tt data}, {\tt capacity}, {\tt reserve}), vector\textless bool\textgreater,
queue ({\tt push}, {\tt pop}), stack ({\tt top}, {\tt push}, {\tt pop}),
%
associative container ({\tt empty}, {\tt size}, {\tt find}, {\tt lower\_bound}, {\tt upper\_bound},
{\tt count}, {\tt emplace}, {\tt insert}, {\tt erase}, {\tt clear}),
pair, map (piecewise\_construct, operator[ ], {\tt at} ({\tt const})), unordered\_map, set, unordered\_set,
%
bitset,
%
shared\_ptr ({\tt make\_shared}, {\tt get}, {\tt reset}, {\tt swap}), unique\_ptr,
lvalue, rvalue, rvalue reference, {\tt move}, {\tt ref}, move constructor, move assignment operator,
%
lambda expression, std::function, {\tt strtol}, {\tt getline},
typename, {\tt nullptr}, {\tt noexcept}, {\tt override},
%
thread (join, detach),
mutex, unique\_lock, condition\_variable, lock\_guard,
atomic operations (atomic\_bool (load, operator=), atomic\_flag), memory ordering, modification order, happens-before, synchronizes-with,
%
postponement, RAII, exception-safe (copy and swap)),
%
Java, Objective-C,
%
Python (sequence type ({\tt len}), list ({\tt remove}), tuple, mapping type, dictionary ({\tt items}),
enumerate, natsorted, sorted, map, filter, lambda, isinstance, ternary, not, in, import, from, parameter passing,
global, nested function, scoping, assert, hex, format,
except, finally, module, argparse, pickle, re, subprocess, binascii, click),
%
Jinja2 (template, custom filter, macro, include, filter (combine)), YAML (sequence, mapping, scalar),
XML, AMPL, MATLAB,
%
script (bash, tcsh, expansion (variable, arithmetic), command substitution, subshell, process substitution \textless(), {\tt read},
lua (and, or, ternary operator)),
Fortran, Verilog, VHDL, SQL %

\item[Operating Systems:] Linux (CentOS, Fedora, Red Hat, Ubuntu, Debian, Raspbian Lite), FreeRTOS, Windows, Windows Subsystem for Linux (WSL), Mac OS X

\item[Networking:] IEEE 802.3, global pause, auto-negotiation, EEE, EPON (802.3ah), 10GEPON (802.3av), flow control (802.3x), LAG (802.3ad),
IEEE 802.1, bridge (.1D, .1Q), VLAN, PFC (802.1Qbb), Q-in-Q, GRE, VXLAN,
control plane (ioctl, AXI4-Lite, Avalon-MM, MDIO (802.3 Clause 22, 45), I$^2$C),
network device driver, ARP (probe, announcement, gratuitous, proxy, RFC 5227), IPv4, LLA (RFC 3927), DSCP, ECN marking (RFC 3168), WRED dropping,
ICMP (RFC 792, echo, echo reply),
TCP, BGP (quagga, frr), UDP, DHCP (RFC 2131), SNMP (agent, ASN.1 encoding/decoding, trap, mib), SNTPv4 (RFC 4330), NTPv4 (RFC 5905),
NicheStack\textsuperscript{TM} IPv4 (arp, udp, snmpv2c, sntpv4), socket programming,
%
InfiniBand, RoCEv1, RoCEv2,
iWARP (RDMAP/DDP/MPA, RFC 5040, 5041, 5044), RDMA, Verbs programming,
%
1588v2,
GigE Vision (WRITEREG, READREG, WRITEMEM, READMEM, EVENTDATA, DISCOVERY, PENDING\_ACK, FORCEIP, heartbeat),
%
Software for Open Networking in the Cloud (SONiC) (SubscriberStateTable, CONFIG\_DB, APPL\_DB,
ProducerTable, ConsumerTable, FLEX\_COUNTER\_DB, ASIC\_DB,
Table, COUNTERS\_DB,
ProducerStateTable, APPL\_DB, ConsumerStateTable, APPL\_DB,
%
NotificationConsumer, APPL\_DB, COUNTERS\_DB, NotificationProducer, ASIC\_DB,
SelectableTimer,
flex counter architecture, sairedis, synchronous SAI api call, PFC storm watchdog,
buffer watermark, buffer config manager, ACL, warm-reboot, scheduler, QoS mapping, neighbor, route),
%
OCP switch abstraction interface (SAI) (pipeline, buffer, Broadcom implementation, python-saithrift rpc library),
FEC (Reed-Solomon)

\item[Switching:] ingress admission control, virtual output queuing, egress admission control,
scheduling (strict priority (preemptive, non-preemptive),
round robin (work-conserving, native (packet), weighted (packet), deficit (byte), weighted deficit (byte)),
policing/traffic shaping (average rate (packet, byte), average rate + burst size (packet, byte), average rate + burst size + peak rate (packet, byte)),
output-queued shared memory, input-output-queued

\item[Linux kernel:] kernel architecture, interrupt context, interrupt \& interrupt handler, softirq, tasklet, kernel synchronization,
RCU, memory barrier, process context, work queue, kernel thread and sleep (wait queue), kernel preemption, hrtimer, memory management,
char driver, network traffic control (TC) (token bucket filter), Tx \& Rx path below layer 3, new API (NAPI),
syscall, sysfs, tracepoint, kprobe, kdump, kbuild, kallsyms

\item[FreeRTOS kernel:] interrupt service routine (ISR), binary semaphore, mutex, queue, event group, task management \& scheduling, software timer,
heap memory management, stack overflow checking

\item[Data structures (program):] array, vector, linked list (singly, doubly), queue, priority queue, stack,
hash table (3sum), lookup table/map/dictionary/associative array, C++ multimap,
tree, binary tree, heap, binary search tree, AVL tree, prefix tree/trie,
bitmap/bit vector/bitset, set (bitmap, array, linked list, hash table, binary search tree),
suffix array

\item[Algorithms:] searching (linear/sequential search, binary search, breadth-first search, depth-first search), iteration, recursion,
sorting (bubble sort, selection sort, insertion sort, shell sort, quick sort, merge sort, radix sort, heap sort),
selection ($K$\textsuperscript{th} smallest/largest),
random sampling (sorted output, random output),
bit manipulation, divide and conquer (quick sort), dynamic programming, greedy, invariant, minimum spanning tree, Dijkstra's algorithm

\item[Applications:] code reading, regular expression (basic, extended, [:space:]), system debugging, Linux kernel programming,
Linux system programming ({\tt select}, {\tt poll}, {\tt epoll}, {\tt getenv}, timerfd), netlink (rt, neigh),
interprocess communication (IPC), process scheduling,
concurrency, multithreading (POSIX), synchronization \& condition synchronization, file access, iOS programming, Android programming,
OPNET simulator, AMPL/lp\_solve, AMPL/CPLEX, Kusto query,
P4 packet test framework (PTF) (dataplane),
redis ({\tt get}, {\tt hget}, {\tt set}, {\tt hset}, {\tt del}, {\tt hdel} (multiple fields), dump, load),
Broadcom SDK (BST, KNET, QoS mapping, FP, flex counters, scheduler, LAG, hashing), OpenNSL,
Azure ExpressRoute

\item[Design patterns:] observer (one to many), bridge (separate an abstraction from its implementation),
decorator (add responsibilities incrementally \& dynamically), singleton, publish-subscribe, composite (composition of objects in tree structure)

\item[Tools:] Makefile (expansion (recursive, simple, secondary), {\tt eval}, phony target), gdb, crash, gprof, OProfile, {\tt iperf},
strace, ftrace, SystemTap,
Scapy (ether, dot1q, ip, arp, icmp), Wireshark, autoconf (./autogen.sh, ./configure, make, make install, fakeroot), cygwin, cross toolchain (OpenWrt, Altera),
JTAG, SnmpB,
{\tt lftp}, {\tt ssh} (localhost), {\tt sftp}, {\tt scp}, SecureCRT (SSH, SFTP), PuTTY,
Jira, CodeFlow,
jumpbox, Dnsmasq, syslog, log analyzer

\item[Linux command line:] {\tt lsblk}, {\tt fdisk}, {\tt df}, {\tt du},
{\tt ps},
{\tt mkfs}, {\tt less}, {\tt tail}, {\tt grep} (-r, -i, -n, -v, -{}-color), {\tt find} (-type), {\tt xargs}, {\tt ln}, {\tt mount}, {\tt unmount}, /etc/fstab,
{\tt file}, {\tt chmod}, {\tt nm},
{\tt tar} (.tar, .tar.gz), {\tt gzip}, {\tt gunzip} (.gz), {\tt zcat}, {\tt zgrep},
{\tt ifconfig}, {\tt ip addr}, {\tt ip link}, {\tt ethtool}, {\tt arp}, {\tt ip neigh}, {\tt ip link}, {\tt tcpdump}, {\tt ip route}, {\tt wget}, {\tt curl} (-o),
{\tt chown}, {\tt disown}, {\tt adduser}, {\tt usermod}, {\tt passwd}, {\tt id},
{\tt dpkg},
{\tt which},
{\tt lsusb},
{\tt virsh},
{\tt diff}, {\tt comm},
{\tt cat} (-A, ctrl-D, \textless\textless token), {\tt awk} (/pattern/action), {\tt cut} (-d, -f, -c),
{\tt sort} (-t, -k, -nr, -u), {\tt uniq} (-u, -d, -c), {\tt sed} (-n),
{\tt date}, {\tt time} (-f)
{\tt top}

\item[Hardware:] NetFPGA-10G, Xilinx, Virtex-5, Altera UDP/IP core, network interface card (Mellanox, Chelsio, Emulex),
switch (Arista 7124SX, 7050QX-32S, 7060CX-32S, 7260CX3-64, Dell S6000, S6100-ON, Mellanox SN2700,
Ubiquiti EdgeSwitch16, Netgear M4200, GS108PE, GS110TP), PCIe, SR-IOV, DMA,
processor microarchitecture (Pentium, Pentium Pro, II, III, PowerPC 601, 603e, 604, 750, 7400),
computer, microcontroller, and SoC architecture (Nios II, Cortex-M3/M4, MicroBlaze, ARM926EJ-S),
exclusive access, memory barrier, write buffer,
switch chip (Broadcom XGS devices, BCM56850 (Trident2), BCM56960 (Tomahawk), BCM56970 (Tomahawk2),
ingress pipeline (L2 unicast \& multicast, L3 unicast \& multicast), egress pipeline (L3),
memory management unit, buffer management, buffer statistics tracking, scheduler,
cross-point (XP) architecture, PFC, ECN, WRED, mirroring (ERSPAN), field processor (legacy, new),
VRF, VXLAN, hashing (flex, resilient), ECMP (weighted, hierarchical),
user defined field (FP, hashing, flex hashing, flexible counter), flexible counters,
BCM88670 (Jericho), BCM88690 (Jericho2), packet processor architecture, traffic management architecture,
Cisco Lacrosse),
MAC core (MorethanIP 10/100/1000Mbps, Xilinx 10G),
PHY (Broadcom AEL2005, Marvell 88X3310, 88E1512, Microsemi VSC8541), Camera Link interface, sensor (Maxim Integrated LM75,
Altera Arria 10 internal temperature sensing diode \& IP core), LED,
FireWire card (SYBA)

\item[Virtualization:] hypervisor (VirtualBox, VMware, Hyper-V, Xen), container, Docker (info,
build context, login, pull, push,
Dockerfile (FROM, RUN, COPY; WORKDIR, EXPOSE, ENTRYPOINT),
load, images, tag, rmi, image (ls, history, inspect, build, tag),
ps, run, exec, stop, start, attach, rm,
cp)

\item[Automation:] Ansible (ansible-playbook, name, with\_items, with\_dict, loop, when, rescue, include, include\_vars, register, become, connection,
async, poll, async\_status, action plugin,
module (set\_fact, debug, fail, copy, template, command, shell, script, supervisorctl, lineinfile), custom module),
Jenkins (template, pipeline)

\item[Test:] pytest, Google test

\item[Version Control:] Git (local, remote, add -{}-patch, branch -{}-contains,
fetch -{}-tags, push -{}-force -{}-delete, show, rebase, apply, cherry-pick, submodule update -{}-recursive, bisect, blame -{}-line-porcelain),
GitHub workflow, Bitbucket, Visual Studio Team Services,
CodeFlow, CVS, TortoiseCVS, SVN, TortoiseSVN, Source Depot

\item[IDE:] Eclipse, Xcode

\item[Text and Graphs:] Vim, GVim, Emacs, Confluence, SharePoint, OneNote, \LaTeX, PSTricks, Gnuplot, Origin, SMIv1, SMIv2, JSON

\item[Garage:] 3D printing (Ultimaker 3 Extended, Cura), laser cutting (Universal laser systems VLS4.60, LibreCAD, AutoCAD, Adobe Illustrator),
plywood, living hinge, safe box, piano, treasure box, ring box, phone stand, drink coaster, anodized aluminum, dog tag,
vinyl cutting (Silhouette Cameo), Raspberry Pi Zero W, Arduino Uno (button, LED, potentiometer, speaker), sewing machine (Brother SE-400)

\end{description}




\resheading{Honors and Awards}%%%%%%%%%%%%%%%%%%%%%%%%%%%%%%%%%%%%%%%%%%%%%%%%%%%%%

\leftskip 0.04in %0.27
\vspace{0.1in}
\begin{tabular*}{7.08in}{l@{\extracolsep{\fill}}r} %6.5in
\textcolor{Red}{\textbf{O-1 Visa---Individuals with Extraordinary Ability or Achievement}} & 2016\\
Outstanding reviewer for Elsevier Optical Switching and Networking & 2016\\
\textcolor{Red}{OCE (Ontario Centres of Excellence) TalentEdge Fellowship for industry R\&D (CAD\$32,500)} & 2015\\
\textcolor{Red}{OCE (Ontario Centres of Excellence) TalentEdge Fellowship for industry R\&D (CAD\$32,500)} & 2014\\
Recognized reviewer for Elsevier Computer Communications & 2014\\
\textcolor{Red}{Runner-up of the 2013 Fabio Neri Best Paper Award} & 2013\\
{Eligibility as NSERC Visiting Fellow in Canadian Government Laboratories} & 2011\\%\textcolor{Red}
{Eligibility as NSERC Visiting Fellow in Canadian Government Laboratories} & 2010\\%\textcolor{Red}
Feng Deng conference travel grant (IEEE/OSA OFC 2010, San Diego, CA, USA, 12,000RMB) & 2009\\
\textcolor{Red}{Semi-finalist of the 2010 Corning Outstanding Student Paper Award (10-12 out of over 430)} & 2009\\
Feng Deng conference travel grant (IEEE/OSA OFC 2009, San Diego, CA, USA, 10,000RMB) & 2008\\
MPS (Monolithic Power System Inc.) Scholarship for excellent students (second class) & 2008\\
\textcolor{Red}{DAAD (Deutscher Akademischer Austausch Dienst) research scholarship} & \multirow{2}{*}{2008}\\ %\officialeuro
\textcolor{Red}{\hfill for young doctoral candidates up to six months (5,000\EURtm)} & \\
Feng Deng conference travel grant (IEEE/OSA OFC 2008, San Diego, CA, USA, 10,000RMB) & 2007\\
Excellent individual in social practice & 2007\\
Feng Deng conference travel grant (IEEE GLOBECOM 2007, Washington D.C., USA, 10,000RMB) & 2007\\
\textcolor{Red}{Excellent graduate of Tsinghua University, an honor entitled to top 10\% graduates} & 2005\\
Shunde Wu Couple Scholarship for excellent students (second class) & 2004\\
Shunde Wu Couple Scholarship for excellent students (second class) & 2003\\
Individual prize in electronics process technology practice & 2003\\
Citic Bank Scholarship for excellent students (second class) & 2002\\
\end{tabular*}




\leftskip 0.0cm
\vspace{0.1in}
\resheading{Professional and Volunteer Activities}%%%%%%%%%%%%%%%%%%%%%%%%%%%%%%%%%%%%%%%%%%%%%%%%%%%%%%
\begin{itemize}
\setlength{\itemindent}{-0.075in}

%\item
%\textbf{Member} of Optical Society of America (OSA), IEEE, and IEEE communications society

\item
Optical Society of America (OSA) Young Professionals (YP) Program, April 2011 - December 2013

\item
\textbf{Technical program committee member} of OSA Photonic Networks and Devices (Networks) 2013, 2014, 2015, 2016, 2017, 2018, 2019

\item
\textbf{Technical program committee member} of IEEE Global Communications Conference (Globecom), Optical Networks and Systems Symposium 2015, 2016, 2017,
2018, 2019

\item
\textbf{Technical program committee member} of IEEE International Conference on Communications (ICC), Optical Networks and Systems Symposium
2012, 2013, 2014, 2015, 2016, 2017, 2019

\item
\textbf{Technical program committee member} of IEEE International Conference on Optical Network Design and Modeling (ONDM)
2013, 2014, 2015, 2016, 2017, 2018, 2019

\item
\textbf{Technical program committee member} of IEEE International Conference on Computing, Networking and Communications (ICNC),
Optical and Grid Computing Symposium 2014, 2015, 2016, 2017, 2018, 2019, 2020

\item
\textbf{Technical program committee member} of OSA Asia Communications and Photonics Conference (ACP),
Track 3 Network Architectures, Management and Applications 2016, 2018

\item
\textbf{Technical program committee member} of IEEE Asia-Pacific Conference on Communications (APCC) 2016

\item
\textbf{Technical program committee member} of IEEE International Conference on Networks (ICON) 2012, 2013

\item
\textbf{Organization committee member} of the 56th Canadian Operational Research Society (CORS) Annual Conference, Telecommunications Cluster 2014

\item
\textbf{Industrial membership committee member} of Asia-Pacific Signal and Information Processing Association (APSIPA) 2018-2019

\item
\textbf{Volunteer} (paper curator, dispatcher, and reviewer; on-site assistant) of the Technical Conference on Linux Networking 2015 (Netdev 0.1)

\item
\textbf{Reviewer} of IEEE/ACM Transactions on Networking (1), IEEE/OSA Journal of Lightwave Technology (6),
IEEE/OSA Journal of Optical Communications and Networking (14), IEEE Transactions on Communications (1), IEEE Communications Letters (1),
Elsevier Computer Networks (3), Elsevier Computer Communications (1), Elsevier Optical Switching and Networking (13),
Springer Photonic Network Communications (7), Springer Telecommunication Systems (3), Wiley International Journal of Communication Systems (1),
Wiley Networks Journal (1), IEEE China Communications (1), De Gruyter Journal of Optical Communications (1), Chinese Optics Letters (1),
Acta Electronica Sinica (2), Journal of Beijing University of Posts and Telecommunications (4),
and many good conferences such as IEEE INFOCOM, IEEE ICC, IEEE GLOBECOM, IEEE DRCN, IEEE ONDM, etc.

\end{itemize}




%\vspace{0.1in}
\resheading{Immigration Status}%%%%%%%%%%%%%%%%%%%%%%%%%%%%%%%%%%%%%%%%%%%%%%%%%%%%%%%%%%%
\vspace{0.15in} %

\leftskip 0.2in %

Permanent resident of Canada, landed on 8 April, 2014

Citizen of Canada, 3 August, 2018

\leftskip 0.0in




\vspace{0.1in}
\resheading{Miscellaneous}
\begin{description}
\setlength{\itemindent}{-0.1in}
\item[GitHub:] https://github.com/wendani

\item[Google Scholar:] https://scholar.google.ca/citations?user=KX0P6HsAAAAJ\&hl=en
\end{description}




%\vspace{0.1in}
\resheading{Patents}%%%%%%%%%%%%%%%%%%%%%%%%%%%%%%%%%%%%%%%%%%%%%%%%%%%%%%%%%%%% (Selected)
\begin{enumerate}
\item
Yunqu Liu, Kin-Wai Leong, \textbf{Wenda Ni}, and Changcheng Huang, ``Methods and systems for passive optical switching,'' publication no. WO/2014/078940, published 30 May, 2014.

\item
Yunqu Liu, Kin-Wai Leong, \textbf{Wenda Ni}, and Changcheng Huang, ``Parallel optoelectronic network that supports a no-packet-loss signaling system and loosely coupled application-weighted routing,'' U.S. patent pending, application no. 61/992,570, filed 13 May, 2014.
\end{enumerate}




\resheading{Journal Papers}%%%%%%%%%%%%%%%%%%%%%%%%%%%%%%%%%%%%%%%%%%%%%%%%%%%%%%%%%%%% (Selected)
\begin{enumerate}
%\setlength{\itemindent}{-0.05in}

\item
\textbf{Wenda Ni}, Changcheng Huang, and Jing Wu, ``Integrated design of fault localization and survivable mapping in IP over transparent WDM networks,'' \textit{Springer Photonic Network Communications}, vol. 28, no. 3, pp. 287--294, 2014.

\item
\textbf{Wenda Ni}, Changcheng Huang, Yunqu Leon Liu, Weiwei Li, Kin-Wai Leong, and Jing Wu, ``POXN: a new passive optical cross-connection network for low-cost power-efficient datacenters,'' \textit{IEEE/OSA Journal of Lightwave Technology}, vol. 32, no. 8, pp. 1482--1500, Apr. 15, 2014.
\textcolor{Red}{(First-round review decision is ``accept pending minor revisions''.)}

\item
\textbf{Wenda Ni}, Changcheng Huang, and Jing Wu, ``Provisioning high-availability datacenter networks for full bandwidth communication,'' \textit{Elsevier Computer Networks, Special Issue on Communications and Networking in the Cloud}, vol. 68, pp. 71--94, Aug. 2014.
(A thorough study with 24 pages in double-column format. 12 papers were accepted out of over 60 submissions. Decision from the first-round review is ``conditional accept'' subject to ``moderate revisions''.)
% \textcolor{Red}

\item
\textbf{Wenda Ni}, Changcheng Huang, Jing Wu, and Michel Savoie, ``Availability of survivable Valiant load balancing (VLB) networks over optical networks,'' \textit{Elsevier Optical Switching and Networking, Special Section on Cross-Layer Innovations}, vol. 10, no. 3, pp. 274--289, Jul. 2013. \textcolor{Red}{(Runner-up of the 2013 Fabio Neri Best Paper Award---one winner and three runners-up were selected from all papers published in this journal in year 2013)}

\item
\textbf{Wenda Ni}, Jing Wu, Changcheng Huang, and Michel Savoie, ``Analytical models of flow availability in two-layer networks with dedicated path protection,'' \textit{Elsevier Optical Switching and Networking, Special Issue on Advances in Optical Networks Control and Management}, vol. 10, no. 1, pp. 62--76, Jan. 2013.

\item
%\textcolor{NavyBlue} {%Midnight
\textbf{Wenda Ni}, Xiaoping Zheng, Chunlei Zhu, Yanhe Li, Yili Guo, and Hanyi Zhang, ``Achieving resource-efficient survivable provisioning in service differentiated WDM mesh networks,'' \textit{IEEE/OSA Journal of Lightwave Technology}, vol. 26, no. 16, pp. 2831--2839, Aug. 15, 2008.
%\textcolor{red} {(the first work made to this top journal by a Ph.D. student/candicate in the laboratory history)}%
%}

\item
Yanwei Li, \textbf{Wenda Ni}, Heng Zhang, Yanhe Li, and Xiaoping Zheng,``Availability analytical model for permanent dedicated path protection in WDM networks,'' \textit{IEEE Communications Letters}, vol. 16, no. 1, pp. 95--97, Jan. 2012.

\item
Qingshan Li, \textbf{Wenda Ni}, Yanhe Li, Yili Guo, Xiaoping Zheng, and Hanyi Zhang, ``Incremental survivable network design with topology augmentation in SDH/SONET mesh networks,'' \textit{Springer Photonic Network Communications}, vol. 18, no. 3, pp. 400--408,
2009.

\item
Chunlei Zhu, \textbf{Wenda Ni}, Yu Du, Yanhe Li, Xiaoping Zheng, Yili Guo, and Hanyi Zhang, ``New and improved approaches for wavelength assignment in wavelength-continuous optical burst switching (OBS) networks,'' \textit{SPIE Optical Engineering}, vol. 46, no. 9, 090504, Sept. 2007.

\item
Qingshan Li, Xiaoping Zheng, \textbf{Wenda Ni}, Yanhe Li, Hanyi Zhang, ``Incremental survivable network design against node failure in SDH/SONET mesh networks," \textit{Springer Photonic Network Communications}, vol. 23, no. 1, pp. 25--32, Feb. 2012.

\item
\textbf{Wenda Ni}, Qingshan Li, Yanhe Li, Hanyi Zhang, Bingkun Zhou, and Xiaoping Zheng, ``Survivability in optical transport networks,'' \textit{Acta Electronica Sinica}, vol. 41, no. 7, pp. 1395--1405, Jul. 2013. (in Chinese)

\item
\textbf{Wenda Ni}, Chunlei Zhu, and Xiaoping Zheng, ``Design and implementation of all optical networks supporting differentiated services,'' \textit{ZTE Communications}, vol. 12, no. 6, pp. 10--13, Dec. 2006. (in Chinese)
\end{enumerate}




\resheading{Conference Papers (Selected)}%%%%%%%%%%%%%%%%%%%%%%%%%%%%%%%%%%%%%%%%%%%%%%%%%%%%%%%%% (Selected)

\begin{enumerate}
%\setlength{\itemindent}{-0.05in}
\item
\textbf{Wenda Ni}, Changcheng Huang, and Jing Wu, ``On capacity provisioning in datacenter networks for full bandwidth communication,'' in \textit{Proc. IEEE International Conference on High Performance Switching and Routing (HPSR)}, July 2013, pp. 62--67.

\item
\textbf{Wenda Ni}, Jing Wu, Changcheng Huang, and Michel Savoie, ``Flow availability analysis in two-layer networks with dedicated path protection at the upper layer,'' in \textit{Proc. IEEE International Conference on Communications (ICC)}, Jun. 2012, CQR-P2.

\item
\textbf{Wenda Ni}, Changcheng Huang, Jing Wu, Qingshan Li, and Michel Savoie, ``Optimizing the monitoring path design for independent dual failures,'' in \textit{Proc. IEEE International Conference on Communications (ICC)}, Jun. 2012, ONS03.

\item
\textbf{Wenda Ni}, Erwin Patzak, Michael Schlosser, and Hanyi Zhang, ``Availability evaluation in shared-path-protected WDM networks with startup-failure-driven backup path reprovisioning,'' in \textit{Proc. IEEE International Conference on Communications (ICC)}, May 2010, ON02.

\item
\textbf{Wenda Ni}, Yabin Ye, Michael Schlosser, Erwin Patzak, and Hanyi Zhang, ``Survivable mapping with maximal physical-layer failure-localization potential in IP over transparent optical networks,'' in \textit{Proc. IEEE/OSA Optical Fiber Communication Conference and Exposition (OFC)}, Mar. 2010, OWH1. \textcolor{Red}{(Semi-finalist in the 2010 Corning Outstanding Student Paper Competition---only 10-12 papers were selected as semi-finalists out of over 430 student submissions)}

\item
\textbf{Wenda Ni}, Erwin Patzak, Michael Schlosser, Yabin Ye, and Hanyi Zhang, ``On operating shared-path-protected WDM networks non-revertively by using backup path reprovisioning,'' in \textit{Proc. IEEE/OSA Optical Fiber Communication Conference and Exposition (OFC)}, Mar. 2010, OWH4.

\item
%\textcolor{NavyBlue} { %
\textbf{Wenda Ni}, Chunlei Zhu, Yabin Ye, Michael Schlosser, and Hanyi Zhang, ``Reducing burst loss probability in multi-class optical burst switching networks by successive minimal incremental routing,'' in \textit{Proc. IEEE/OSA Optical Fiber Communication Conference and Exposition (OFC)}, Mar. 2009, OWA6.
%\textcolor{NavyBlue} {(tier 1
%conference in optical communications)}%
%}

\item
%\textcolor{NavyBlue} {% Midnight
\textbf{Wenda Ni}, Michael Schlosser, Qingshan Li, Yili Guo, Hanyi Zhang, and Xiaoping Zheng, ``Achieving optimal lightpath scheduling in survivable WDM mesh networks,'' in \textit{Proc. IEEE/OSA Optical Fiber Communication Conference and Exposition (OFC)}, Feb. 2008, OWN5.
%\textcolor{NavyBlue} {(tier 1
%conference in optical communications)}%
%}

\item
%\textcolor{NavyBlue} {% Midnight
\textbf{Wenda Ni}, Chunlei Zhu, Xiaoping Zheng, Yanhe Li, Yili Guo, and Hanyi Zhang, ``On routing optimization in multi-class optical burst switching networks,'' in \textit{Proc. IEEE International Conference on Communications (ICC)}, May 2008, ON03-4.
%\textcolor{NavyBlue} {(prestige conference
%of IEEE communications society)}%
%}

\item
%\textcolor{NavyBlue} {% Midnight
\textbf{Wenda Ni}, Xiaoping Zheng, Chunlei Zhu, Yili Guo, Yanhe Li, and Hanyi Zhang, ``An improved approach for online backup reprovisioning against double near-simultaneous link failures in survivable WDM mesh networks,'' in \textit{Proc. IEEE Global Communications Conference (GLOBECOM)}, Nov. 2007, ONSS05-2.
%\textcolor{NavyBlue} {(prestige conference of
%IEEE communications society)}%
%}

\item
Yanwei Li, \textbf{Wenda Ni}, Heng Zhang, Nan Hua, Yanhe Li, and Xiaoping Zheng, ``Availability analytical model for permanent dedicated path protection in service differentiated WDM networks,'' in \textit{Proc. IEEE/OSA Optical Fiber Communication Conference and Exposition (OFC)}, Mar. 2013, JW2A.01.

\item
Jing Wu, \textbf{Wenda Ni}, and Changcheng Huang, ``Flow availability in two-layer networks with dedicated path protection,'' in \textit{Proc. OSA/IEEE/SPIE Asia Communications and Photonics Conference (ACP)}, Nov. 2012, ATh4D. (Invited talk)

%\item
%Yanwei Li, \textbf{Wenda Ni}, Yanhe Li, and Xiaoping Zheng, ``Availability analysis of permanent dedicated path protection in WDM mesh networks,'' in \textit{Proc. OSA/IEEE/SPIE Asia Communications and Photonics Conference and Exhibition (ACP)}, Dec. 2010, FD3.
%
%\item
%Qingshan Li, \textbf{Wenda Ni}, Yanhe Li, Hanyi Zhang, and Xiaoping Zheng, ``Incremental network design with topology augmentation on backup path provisioning in WDM mesh networks,'' in \textit{Proc. OSA/IEEE/SPIE Asia Communications and Photonics Conference and Exhibition (ACP)}, Dec. 2010, P79.
%
%\item
%\textbf{Wenda Ni}, and Hanyi Zhang, ``ILP-based study on blocking-event-triggered backup path reoptimization under the dynamic resource-efficient provisioning framework,'' in \textit{Proc. 8th International Conference on Optical Internet (COIN)}, Nov. 2009, 5.
%
%\item
%\textbf{Wenda Ni}, Michael Schlosser, Hanyi Zhang, and Erwin Patzak, ``Blocking-differentiated path provisioning in semi-dynamic survivable WDM networks,'' in \textit{Proc. OSA/IEEE/SPIE Asia Communications and Photonics Conference and Exhibition (ACP)}, Nov. 2009, WF4.
%
%\item
%Qingshan Li, \textbf{Wenda Ni}, Yanhe Li, Yili Guo, Hanyi Zhang, and Xiaoping Zheng, ``Improving the dual-failure restorability in scheduled WDM mesh networks,'' in \textit{Proc. OSA/IEEE/SPIE Asia Communications and Photonics Conference and Exhibition (ACP)}, Nov. 2009, WF2.
%
%\item
%\textbf{Wenda Ni}, Michael Schlosser, and Hanyi Zhang, ``Backup path reprovisioning and activation planning with differentiated dual-failure restorability in WDM mesh networks,'' in \textit{Proc. VDE ITG-Fachtagung Photonische Netze}, May 2009.
%
%\item
%\textbf{Wenda Ni}, Qingshan Li, Yabin Ye, Yili Guo, Hanyi Zhang, and Xiaoping Zheng, ``On backup path activation to achieve differentiated dual-failure restorability in WDM mesh networks,'' in \textit{Proc. IEICE 7th International Conference on Optical Internet (COIN)}, Oct. 2008, 11.
%
%\item
%\textbf{Wenda Ni}, Qingshan Li, Yabin Ye, Yanhe Li, Yili Guo, Hanyi Zhang, and Xiaoping Zheng, ``Performance evaluation of a resource-efficient provisioning framework (RPF) in service differentiated survivable WDM networks subject to wavelength-continuity constraint,'' in \textit{Proc. SPIE Asia-Pacific Optical Communications (APOC)}, Oct. 2008, 7137-44.
%
%\item
%Qingshan Li, \textbf{Wenda Ni}, Yabin Ye, Yanhe Li, Yili Guo, Hangyi Zhang, and Xiaoping Zheng, ``On lightpath scheduling in service differentiated survivable WDM mesh networks,'' in \textit{Proc. SPIE Asia-Pacific Optical Communications (APOC)}, Oct. 2008, 7137-112.
%
%\item
%Meng Wu, \textbf{Wenda Ni}, Yabin Ye, Yili Guo, and Hanyi Zhang, ``Capacity allocation for time-varying traffic in survivable WDM mesh networks,'' in \textit{Proc. SPIE Asia-Pacific Optical Communications (APOC)}, Oct. 2008, 7137-16.
%
%\item
%\textbf{Wenda Ni}, Chunlei Zhu, Xiaoping Zheng, Yanhe Li, Yili Guo, and Hanyi Zhang, ``On differentiated service provisioning in survivable WDM mesh networks,'' in \textit{Proc. SPIE Asia-Pacific Optical Communications (APOC)}, vol. 6487, Nov. 2007, 6487-118.
%
%\item
%\textbf{Wenda Ni}, Xin Yu, Xiaoping Zheng, and Yanhe Li, ``Load balancing method for constraint-based wavelength routing in service-guaranteed optical networks,'' in \textit{Proc. SPIE Asia-Pacific Optical Communications (APOC)}, vol. 6022, Nov. 2005, 602206.

\end{enumerate}







%\resheading{Biography}%%%%%%%%%%%%%%%%%%%%%%%%%%%%%%%%%%%%%%%%%%%%%%%%%%%%%%%%%%%%%%%%%%%%%%%%%%%
%\vspace{0.15in} %
%Wenda Ni received the B.E. degree (with excellency) in electronic
%science and technology in 2005 from Tsinghua University, P. R.
%China, where he is currently working towards the Ph.D. degree with
%the Department of Electronic Engineering.
%
%From Feb. 2005 to July 2010, he is a research assistant with the
%Laboratory of Optical Networking and Microwave Photonics at Tsinghua
%University, where he is involved in several research projects. From
%Oct. 2008 to Mar. 2009, being granted the DAAD research fellowship,
%he is a visiting student with the network design and modeling group,
%Department of Photonic Networks and Systems (PN),
%Fraunhofer-Institute for Telecommunications,
%Heinrich-Hertz-Institut, Berlin, Germany, under the supervision of
%Dr. Erwin Patzak. His research interests include network design,
%operations and management with a focus on survivability, service
%differentiation, and multi-layer traffic engineering in WDM mesh
%networks (backbone networks).
%
% Mr. Ni is a student member of IEEE and OSA.%
%\vspace{0.1in} %\vspace{0.5in}

%\newpage
\resheading{References}%%%%%%%%%%%%%%%%%%%%%%%%%%%%%%%%%%%%%%%%%%%%%%%%%%%%%%%%%%%%%%%%%%%%%%%%%%%

\leftskip 0.2in %

%\vspace{0.2in}%
%Available upon request

%\vspace{0.2in}%
%\textbf{Chunming Qiao, Ph.D., Fellow, IEEE}\\
%\textit{Professor}, Department of Computer Science and Engineering\\
%State University of New York at Buffalo\\
%Tel.: +1 (716) 645 3180 ext. 140\\
%E-mail: qiao@cse.buffalo.edu\\
%URL: http://www.cse.buffalo.edu/\~{}qiao
%
%
%\newpage
\vspace{0.2in}%
\textbf{Changcheng Huang, Ph.D., P. Eng.}\\
\textit{Professor}, Department of Systems and Computer Engineering,
Carleton University\\
Tel: +1 (613) 260 7387\\
E-mail: huang@sce.carleton.ca\\
URL: \url{http://www.sce.carleton.ca/faculty/huang.html}

%\vspace{0.2in}%
%\textbf{Jing Wu, Ph.D.}\\
%\textit{Research Scientist}, Communications Research Centre (CRC) Canada\\
%\textit{Adjunct Professor}, School of Electrical Engineering and Computer Science, University of Ottawa\\
%Tel: +1 (613) 998 2474\\
%E-mail: jingwu@ieee.org\\
%URL: \url{http://www.site.uottawa.ca/~jingwu/}



%%%%%%%%%%%%%%%%%%%%%%%%%%%%%%%%%
\vspace{0.2in}%
%\newpage
\textbf{Daniel Wong}\\
\textit{VP Product}, Viscore Technologies Inc.\\
7 Bayview Road\\
Ottawa, ON K1Y 2C5, Canada\\
Tel: +1 (613) 252 7388\\
%Fax: +86-10-6277-0317\\
E-mail: danielyhwong@gmail.com\\




%%%%%%%%%%%%%%%%%%%%%%%%%%%%%%%%%
\vspace{0.2in}%
%\newpage
\textbf{Victor Yu Liu, Ph.D.}\\
\textit{Chief Network Architect}, Visa\\
Foster City, CA, USA\\
Tel: +1 (510) 384 3811\\
%Fax: +86-10-6277-0317\\
E-mail: viliu@visa.com, packerliu@gmail.com, yuliu@ieee.org\\
URL: \url{http://www.sis.pitt.edu/~yliu/}


%%%%%%%%%%%%%%%%%%%%%%%%%%%%%%%%%
\vspace{0.2in}%
\textbf{Hanyi Zhang}\\
\textit{Past head}, Laboratory of Optical Networking and Microwave
Photonics\\
\textit{Professor}, Department of Electronic Engineering\\
Tsinghua University\\
%Room 11-311, East Main Building\\
%100084 Beijing\\
%P. R. China \\
%Tel: +86 (10) 6277 3377, Fax: +86 (10) 6277 0317\\
E-mail: zhy-dee@tsinghua.edu.cn\\
%URL: http://www.ee.tsinghua.edu.cn/\~{}zhanghanyi/indexe.htm% \url{}


%%%%%%%%%%%%%%%%%%%%%%%%%%%%%%%%%
%\newpage
\vspace{0.2in}%
\textbf{Michael Schlosser}\\
%\textit{Leader},  ``Network design and modeling'' group\\
Project Manager\\
Department of ``Photonic Networks and Systems (PN)''\\
Fraunhofer-Institute for Telecommunications, Heinrich-Hertz-Institut\\
Einsteinufer 37, D-10587 Berlin, Germany\\
Tel: +49 (30) 31002 346, Fax: +49 (30) 31002 250\\
E-mail: michael.schlosser@hhi.fraunhofer.de\\ %\url{}


%%%%%%%%%%%%%%%%%%%%%%%%%%%%%%%%
%\newpage
\vspace{0.2in}%
\textbf{Yabin Ye, Ph.D.}\\
\textit{Senior Researcher}, European Research Center, Huawei Technologies\\
Riesstr. 25, 80992 Munich, Germany\\
Tel: +49 (89) 158 834 4052\\
E-mail: yeyabin@huawei.com; yabin.ye@ieee.org\\
%yabin.ye@huawei.com\\ %\url{}


%%%%%%%%%%%%%%%%%%%%%%%%%%%%%%%%
%\newpage
\vspace{0.2in}%
\textbf{Robert Xin Liu}\\
\textit{CEO}, Source Technologies International, Ltd.\\
Room 102, Unit 3, Building No. 1, Section 3, Shang He Cun, Haidian District\\
Beijing 100097, China\\
Tel: +86 (10) 8849 3060; +86 137 0119 4002\\
E-mail: robert.liu@srctek.com\\



%%Xiaoping Zheng Professor, Department of Electronic Engineering,
%%Tsinghua University Room 11-305, East Main Building Beijing, 100084,
%%P. R. China Tel: +86-10-6277-2670, Fax: +86-10-6277-0317 E-mail:
%%xpzheng@tsinghua.edu.cn
%%http://thulonmp.org/english/english/teacher/zhengxiaoping.html
%
%%\vspace{0.2in}%
%%\textbf{Yili Guo}\\
%%\textit{Professor (retired)}, Department of Electronic Engineering,
%%Tsinghua University\\
%%Room 11-311, East Main Building\\
%%100084 Beijing, P. R. China\\
%%Tel: +86-10-6277-3377, Fax: +86-10-6277-0317\\
%%E-mail: gyl-dee@tsinghua.edu.cn\\ %\url{}
%%URL: http://thulonmp.org/english/english/teacher/guoyili.html
%

%%%%%%%%%%%%%%%%%%%%%%%%%%%%%%%%%%
%%\vspace{0.2in}%
%%\textbf{Dr. Gangxiang Shen}\\
%%\textit{Lead Engineer}, Ciena Corporation\\
%%Linthicum MD 21090\\
%%USA\\
%%%Tel: +86-10-6277-3377, Fax: +86-10-6277-0317\\
%%E-mail: egxshen@gmail.com; gshen@ciena.com\\ %\url{}
%%URL: http://www.gangxiang-shen.com




%\resheading{English Proficiency}%%%%%%%%%%%%%%%%%%%%%%%%%%%%%%%%%%%%%%%%%%%%%%%%%%%%%%%%%%

%\leftskip 0.2in

%\vspace{0.15in} %
%\textbf{TOEFL(IBT):} Reading: 29, Listening: 28, Speaking: 20, Writing: 25, Total: 102\\ %
%\textbf{IELTS Academic:} Listening: 7.5, Reading: 8.0, Writing: 6.0, Speaking: 6.0, Overall: 7.0\\ %
%\textbf{IELTS General Training:} Listening: 8.0, Reading: 8.5, Writing: 7.5, Speaking: 6.0, Overall: 7.5\\ %
%\textbf{GRE:} Verbal: 540, Quantitative: 800, Analytical writing: 4.0\\ %


%\begin{tabular*}{6.5in}{l@{\extracolsep{\fill}}r}
%TOEFL: Reading: 29, Listening: 28, Speaking: 20, Writing: 25, Total: 102\\ %
%IELTS: Listening: 7.5, Reading: 8.0, Writing: 6.0, Speaking: 6.0, Overall: 7.0\\ %
%GRE: Verbal: 540, Quantitative: 800, Analytical writing: 4.0\\ %
%Public English Test System Level 5 (PETS-5): Pass\\ %
%English Proficiency Test II of Tsinghua University (TEPT-II):  Pass\\ %
%English Proficiency Test I of Tsinghua University (TEPT-I):  Excellent\\ %
%College English Test-6 (CET-6):  Pass\\ %
%College English Test-4 (CET-4):  Excellent\\
%\end{tabular*}

%\textbf{Public English Test System Level 5 (PETS-5):} Pass\\ %
%\textbf{English Proficiency Test II of Tsinghua University (TEPT-II):}  Pass\\ %
%\textbf{English Proficiency Test I of Tsinghua University (TEPT-I):}  Excellent\\ %
%\textbf{College English Test-6 (CET-6):}  Pass\\ %
%\textbf{College English Test-4 (CET-4):}  Excellent\\

%\leftskip 0.0cm \vspace{0.1in}








%\resheading{Work Experience}%%%%%%%%%%%%%%%%%%%%%%%%%%%%%%%%%%%%%%%%%%%%%%%%%%%%%%%%%%%%%%%%%%%%%%%%%%%
%\begin{itemize}
%\item
%    \ressubheading{Message Systems, Inc.}{Columbia, MD}{Email Infrastructure Software Engineer}{March 2008 - Present}
%    \begin{itemize}
%        \resitem{Part of a small team developing the Ecelerity mail transport agent.}
%        \resitem{Responsibilities include general development and testing of the MTA as well as release engineering.}
%    \end{itemize}
%
%\item
%    \ressubheading{Tresys Technology, LLC.}{Columbia, MD}{Principal Engineer}{May 2007 - March 2008}
%    \begin{itemize}
%        \resitem{Principal Engineer of the Funded Research \& Development team.}
%        \resitem{Responsible for oversight of multiple projects within the FR\&D group, which specializes in researching techniques to increase the usability of Security Enhanced Linux (SELinux).}
%        \resitem{Provide technical oversight and guidance for research tasks.}
%    \end{itemize}
%
%\item
%    \ressubheading{SPARTA, Inc.}{Columbia, MD}{Principal Engineer}{Sep. 2005 - Mar. 2007}
%    \begin{itemize}
%        \resitem{Led a small team of developers responsible for the production of a security-enhanced version of Apple's Mac OS X operating system, utilizing type enforcement and mandatory access controls.}
%        \resitem{Extended the SELinux FLASK architecture to secure Mach inter-process communication as present in Mac OS X.}
%        \resitem{Extended and enhanced the TrustedBSD MAC Framework for the Darwin kernel, portions of which appear in Mac OS 10.5 (Leopard).}
%    \end{itemize}
%
%\item
%    \ressubheading{Looking Glass Systems, LLC.}{Boulder, CO}{Senior Programmer and System Administrator}{Feb. 2005 - Sep. 2005}
%    \begin{itemize}
%        \resitem{Served as part of a team to design and develop an agent-based monitoring system for Windows and {\sc UNIX} systems.}
%        \resitem{Responsible for the design and implementation of an agent for UNIX-like systems that interoperates with the LG Vision server software.}
%        \resitem{Was also responsible for the installation and maintenance of network and computing resources.}
%    \end{itemize}
%
%\item
%    \ressubheading{GratiSoft, Inc.}{Boulder, CO}{President}{Oct. 2003 - Feb. 2005}
%    \begin{itemize}
%        \resitem{GratiSoft provided commercial support for the Sudo root privilege control package as well as consulting services for OpenBSD and other Open Source software.}
%        % XXX - more detail here too
%    \end{itemize}
%
%\item
%    \ressubheading{Distributed Systems Lab, University of Pennsylvania}{Philadelphia, PA}{Sr. Systems Programmer}{Dec. 2001 - Oct. 2003}
%    \begin{itemize}
%        \resitem{Added KeyNote trust-management support to the Apache web server.}
%        \resitem{Continued to enhance the OpenBSD operating system on a daily basis.}
%        % XXX - list some OpenBSD things I worked on
%    \end{itemize}
%
%\item
%    \ressubheading{Computer Science Operations Group, University of Colorado}{Boulder, CO}{Sr. System Administrator}{Oct. 1993 - Dec. 2001}
%    \begin{itemize}
%        \resitem{One of three full-time {\sc UNIX} system and network administrators in charge of the {\sc UNIX} computing resources for the Computer Science Department.}
%        \resitem{Managed a network of approximately 350 {\sc UNIX} workstations and X-terminals located in undergraduate, masters, and research labs as well as in faculty offices.}
%        \resitem{Responsible for day-to-day operation of department-wide computer resources and computer support.}
%        %Put more stuff back?
%    \end{itemize}
%
%%\pagebreak
%
%\item
%    \ressubheading{Undergraduate Operations Group}{Boulder, CO}{Manager}{Sep. 1992 - Apr. 1993}
%    \begin{itemize}
%        \resitem{Assigned as Manager and Senior System/Network Administrator for a lab of 70 workstations.}
%        \resitem{Supervised four part-time student employees and several student volunteers.}
%        \resitem{Responsible for day-to-day operation of the lab, including user support.}
%    \end{itemize}
%
%\item
%    \ressubheading{{\sc \bf UUNET} Technologies}{Falls Church, VA}{Assistant Postmaster}{May 1992 - Aug. 1992}
%    \begin{itemize}
%        \resitem{Helped administer mail, news, and {\sc UUCP} on Sun SPARC workstations.}
%        \resitem{Wrote a database to track information requests from potential customers.}
%        \resitem{Ported programs from BSD Networking Release 2 to SunOS 4.1.2.}
%        \resitem{Implemented secure versions of Kermit, xmodem, ymodem, and zmodem for {\sc UUNET}'s dial-up software archive.}
%        %\resitem{Extended the "runas" program to support command line argument rewriting.}
%    \end{itemize}
%
%\item
%    \ressubheading{Undergraduate Computer Lab, University of Colorado}{Boulder, CO}{System Administrator}{Jan. 1991 - Apr. 1992}
%    \begin{itemize}
%        \resitem{Responsibilities included hardware and software installations, network troubleshooting, and user support.}
%        \resitem{Assisted in the administration of the Computer Science Department's research network of {\sc UNIX} workstations.}
%    \end{itemize}
%
%\end{itemize}
%

%\resheading{Open Source Projects}%%%%%%%%%%%%%%%%%%%%%%%%%%%%%%%%%%%%%%%%%%%%%%%%%%%%%%%%%%%%%%%%%%%%%%%%%%%
%
%\begin{description}
%\item[2007--Present] One of four upstream maintainers of the SELinux tool chain.
%\item[2001--Present] Major contributor to ISC cron (formerly Vixie cron).
%\item[1996--Present] Core member of the OpenBSD operating system project.  Participated in multiple security audits of the OpenBSD code base.  Responsible for the OpenBSD C library and large portions of the OpenBSD user space.
%\item[1993--Present] Lead developer of the \emph{Sudo} root privilege control package.
%\item[1993--Present] Contributor to other various and sundry Open Source projects.
%\end{description}



\end{document}
