%%%%%%%%%%%%%%%%%%%%%%%%%%%%%%%%%%%%%%%%%
% Medium Length Professional CV
% LaTeX Template
% Version 2.0 (8/5/13)
%
% This template has been downloaded from:
% http://www.LaTeXTemplates.com
%
% Original author:
% Trey Hunner (http://www.treyhunner.com/)
%
% Important note:
% This template requires the resume.cls file to be in the same directory as the
% .tex file. The resume.cls file provides the resume style used for structuring the
% document.
%
%%%%%%%%%%%%%%%%%%%%%%%%%%%%%%%%%%%%%%%%%

%----------------------------------------------------------------------------------------
%	PACKAGES AND OTHER DOCUMENT CONFIGURATIONS
%----------------------------------------------------------------------------------------

\documentclass{resume} % Use the custom resume.cls style

\usepackage[left=0.75in,top=0.6in,right=0.75in,bottom=0.6in]{geometry} % Document margins

\name{Wenda Ni, Ph.D.} % Your name
\address{29-3240 Southgate RD, Ottawa, ON K1V 8W7, Canada} % Your address
\address{M: +1 (613) 790 6788 E: wonda.ni@gmail.com (primary), wenda.ni@viscore.com} % Your phone number and email

\begin{document}

%----------------------------------------------------------------------------------------
%	OBJECTIVE SECTION
%----------------------------------------------------------------------------------------

\begin{rSection}{Objective}
Kernel and system level software design and development for networking and embedded systems
\end{rSection}

%----------------------------------------------------------------------------------------
%	WORK EXPERIENCE SECTION
%----------------------------------------------------------------------------------------

\begin{rSection}{Work Experience}

\begin{rSubsection}{Viscore Technologies, Inc}{Ottawa, ON, Canada}{Lead R\&D engineer on network architecture, software, and hardware}{Sept. 2014 - Present}
\item Architected the integration and interface solution of embedding our own link-layer protocol---HEDA---in the Linux networking system.
\item Designed and currently implementing software modules of HEDA at the kernel device driver level.
\item Designed and implemented a user-space token-passing protocol to demonstrate burst data transmission between two hosts connected through an optical coupler.
\item Developed in-depth knowledge of NIC hardware data plane and control plane architecture. Disabled link fault signaling at the physical layer to allow burst-mode transmission.
\end{rSubsection}

%------------------------------------------------

\begin{rSubsection}{Source Technologies International, Ltd.}{Ottawa, ON, Canada}{Consulting software engineer}{Jun. 2014 - Aug. 2014}
\item Developed a mobile scenario manager app---a commercial product for iPhone iOS---in 1.5 months with both Objective-C and iOS programming learned from scratch.
\item Developed an Android counterpart fulfilling the same requirements.
\end{rSubsection}

\end{rSection}


\begin{rSection}{R\&D Experience}

\begin{rSubsection}{Carleton University}{Ottawa, ON, Canada}{Researcher}{Jun. 2011 - Aug. 2014}
\item Proposed POXN---a datacenter network architecture using passive optical devices. Designed HEDA---a distributed link-layer communication protocol. Granted OCE (Ontario Centres of Excellence) TalentEdge fellowship for research commercialization. % Resolved physical-layer scalability issue.
\item Developed an availability model for Valiant load balancing networks over optical networks. Honored as a runner-up for the 2013 Fabio Neri Best Paper Award (4 papers out of year 2013).
\end{rSubsection}

%------------------------------------------------

\begin{rSubsection}{NEC Laboratories America, Inc.}{Princeton, NJ, USA}{Visiting researcher, Dept. of optical networking}{Jan. 2011 - Feb. 2011}
\item Setup Xen 4.0.1 pv-ops Ubuntu 10.10 through kernel configuration \& building. 
\item Demonstrated VM live migration between two physical hosts connected by a gigabit switch.
\end{rSubsection}

\end{rSection}



%----------------------------------------------------------------------------------------
%	TECHNICAL STRENGTHS SECTION
%----------------------------------------------------------------------------------------

\begin{rSection}{Technical Skills}

\begin{tabular}{ @{} >{\bfseries}l @{\hspace{6ex}} l }
Languages & C, assembly, Java, C++, Objective-C, Python, XML, AMPL, MATLAB, script, \\
& Verilog \\
OS & Linux, Windows, Mac OS X \\
Networking & IEEE 802.3, EPON, IEEE 802.1, bridge, VLAN, PFC, TCP/IP, UDP, socket, \\
& InfiniBand, RoCE, iWARP, RDMA Verbs \\
Linux kernel & architecture, interrupt context, synchronization, process context, hrtimer, \\
& device driver, memory, traffic control, Tx \& Rx path below layer 3, NAPI, \\
& tracepoint, kprobe \\
\end{tabular}

\begin{tabular}{ @{} >{\bfseries}l @{\hspace{6ex}} l }
Applications & Linux kernel programming, Linux system programming, iOS programming, \\
& Android programming, OPNET simulator, design patterns \\
Tools & Makefile, gdb, gprof, Wireshark \\
Hardware & NetFPGA-10G, Xilinx, Virtex-5, NIC (Mellanox, Chelsio, Emulex), \\
& switch (Arista) \\
Version Control & Git \\
IDE & Eclipse, Xcode  \\
Text and Graphs & Vim, Emacs, \LaTeX, PSTricks, Gnuplot
\end{tabular}

\end{rSection}



%----------------------------------------------------------------------------------------
%	EDUCATION SECTION
%----------------------------------------------------------------------------------------

\begin{rSection}{Education}

{\bf Tsinghua University, Beijing, China} \hfill {\em Sept. 2005 - Jul. 2010} \\ 
Ph.D., Electronic Engineering \\
Thesis: Resource optimization and service quality in WDM optical networks \\
DAAD (Deutscher Akademischer Austausch Dienst) research fellowship \\
Semi-finalist for the 2010 Corning Outstanding Student Paper Award (10-12 out of over 430)

{\bf Tsinghua University, Beijing, China} \hfill {\em Sept. 2001 - Jul. 2005} \\ 
B.Eng., Electronic Engineering \\
Major: Physical electronics and Optoelectronics \\
Overall GPA: 88.8/100 Rank: 5/80 \\
Excellent graduate of Tsinghua University, an honor entitled to top 10\% graduates

\end{rSection}



\begin{rSection}{Patents}

Yunqu Liu, Kin-Wai Leong, \textbf{Wenda Ni}, and Changcheng Huang, ``Methods and systems for passive optical switching,'' publication no. WO/2014/078940, published 30 May, 2014.

Yunqu Liu, Kin-Wai Leong, \textbf{Wenda Ni}, and Changcheng Huang, ``Parallel optoelectronic network that supports a no-packet-loss signaling system and loosely coupled application-weighted routing,'' U.S. patent pending, application no. 61/992,570, filed 13 May, 2014.
\end{rSection}



\begin{rSection}{Activities}
Technical program committee (TPC) member for IEEE ICC, GLOBECOM, OSA NETWORKS, etc. \\
Reviewer for IEEE, OSA, Elsevier, Springer, Wiley, etc. \\
Volunteer of the Technical Conference on Linux Networking 2015 (Netdev 0.1)
\end{rSection}


%----------------------------------------------------------------------------------------

\end{document}
